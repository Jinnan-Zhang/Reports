\newcommand{\RNum}[1]{\uppercase\expandafter{\romannumeral #1\relax}}
\chapter{CEPC 上的物理概述}\label{chap:overview}

\section{CEPC 上Z-工厂中的味物理}\label{Sec:2.5}

这样一个能够产生$10^{12} $个Z玻色子的高亮度的Z工厂给味物理测量提供了独一无二的机会。尤其是这$10^{12}$ 个Z玻色子将衰变产生大约$10^{11}$ 个b强子,这比作为B工厂的BaBar和Belle实验产生的B介子的数目高了大约两个数量级,且与Belle \RNum{2}升级后的预期b强子产额相当。

由于此B工厂主要运行于$\Upsilon\ (4S)$ 共振态模式,其主要产生$B^0$和$B^\pm$介子,也能在短时间的$\Upsilon(5S)$共振态运行中产生少量的$B_s$介子。另一方面,运行在Z极点的装置不仅能够产生大量的$B^0$,$B^\pm$和$B_s$介子,还能产生大量的b重子样本。在表$\ref{Table2.4}$中我们比较了$10^{12}$个Z玻色子衰变产生的b强子预期数目与Belle \RNum{2}上的$\Upsilon(4S)$共振模式产生的50 $ab^{-1}$ 和 $\Upsilon(5S)$ 产生的5 $ab^{-1}$ b强子以及LHCb产生的的50 $fb^{-1}$ b强子。CEPC预期的标记效率可以保守地通过LEP的DELPHI合作组的b-标记效率进行估计,粗略估计是30\% ,与Belle\RNum{2}相近。对于Tera-Z我们同样列出了其产生的璨强子和$\tau$ 轻子数目(这里我们使用了已知的Z分支比BR$(Z\to b\bar{b})=(15.12\pm 0.05)\%$,BR(Z$\to c\bar{c})=(12.03\pm 0.21)\%$,BR$(Z\to \tau^+\tau^-)=(3.3696\pm0.0083)\%$ 以及文献中给出的b和c强子化分支比。)利用大量的产生于b/c强子和$\tau$的样例,CEPC上的tera-Z工厂能够研究这些粒子的稀有衰变道,其中很多超越了任何正在运行的或计划中的实验的精度。此外,$10^{12}$个Z玻色子同样能够以前所有为的精度测量Z衰变中的味破坏现象。

\begin{table}[!h]
	\vspace{20pt}
	\centering
	\begin{tabular}{cccc}
		\toprule
		{\bfseries 粒子} &{\bfseries Tera-Z} & {\bfseries Belle \RNum{2}} & {\bfseries LHCb} \\
		\hline
		{\bfseries b 强子} \\
		$B^+$ & $6\times 10^{10}$ & $3\times 10^{10}(50\ ab^{-1}\ on\ \Upsilon(4S))$ & $3\times 10^{13}$ \\
		$B^0$ & $6\times 10^{10}$ & $3\times 10^{10}(50\ ab^{-1}\ on\ \Upsilon(4S))$ & $3\times 10^{13}$ \\
		$B_s$ & $2\times 10^{10}$ & $3\times 10^{8}(5\ ab^{-1}\ on\ \Upsilon(5S))$ & $8\times 10^{12}$ \\
		b 重子  & $1\times 10^{10}$ &  & $1\times 10^{13}$ \\
		$\Lambda^b$  & $1\times 10^{10}$ &  & $1\times 10^{13}$ \\
		{\bfseries c 强子}\\
		$D^0$ & $2\times 10^{11}$ \\
		$D^+$ &  $6\times 10^{10}$ \\
		$D^+_s$ & $3\times 10^{10}$ \\
		$\Lambda^+_c$ & $2\times 10^{10}$ \\
		$\tau^+$ & $3\times 10^{10}$ & $ 5\times 10^{10}(50\ ab^{-1}\ on\ \Upsilon(4S))$ \\
		\bottomrule       
	\end{tabular}
	\caption{Tera-Z上$10^{12}$个Z玻色子衰变产生的粒子预期数目一览。我们使用了文献中的强子化分支比。对那些与此研究有关的衰变,上面同样展示了Belle \RNum{2} 上运行的所有$ 50\ ab^{-1}\ \Upsilon(4S)$ 共振态和$\Upsilon(5S)$ 共振态模式,以及LHCb上的$50\ fb^{-1}$ 的b强子数目(使用了LHCb探测器的$b\bar{b}$ 接受度和文献中的强子化分支比)。 }
	\label{Table2.4}
\end{table}


Z极运行的正负电子对撞机和LHCb实验、Belle \RNum{2} B工厂之间有关键性的不同。与LHCb相比,正负电子对撞机的环境更干净,也即是本底更低。低本底使得利用中性末态粒子例如光子,中性$\pi$介子和K介子进行重建工作成为可能。与Belle \RNum{2} 味工厂相比,运行在Z极使b强子及其衰变产物有大得多的boost。一方面,更大的boost能够使次级顶点的位移更大,且能够使利用$\tau$ 末态进行重建变得更容易。另一方面,更大的boost能够让衰变产物轨迹更平行。这一点可能对于有消失能量的过程的动能的约束有帮助,但也可能使得分离衰变产物变得更具挑战性。从这一方面讲,更深入的关于这些效应和他们如何影响事例挑选后的统计数据的定量的研究将会非常有用。


